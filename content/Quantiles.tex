\section{Quantiles of a Distribution}

Let $\alpha$ in $(0,1)$. The quantile of order $1 - \alpha$ of a random variable $X$ is the number $q_{\alpha}$ such that:\\
\begin{align*}
\displaystyle \mathbb{P}\left(X\leq q_{\alpha }\right)& = 1-\alpha\\
\mathbb{P}(X \geq q_{\alpha })& = \alpha\\
F_X(q_{\alpha})& = 1 - \alpha\\
F^{-1}_{X}(1-\alpha)&= q_{\alpha }
\end{align*}
If the distribution is \textbf{standard normal} $X \sim N(0,1)$:\\
\begin{align*}
\mathbb{P}(|X| > q_{\alpha})& = \alpha\\
& = 2\mathbf{\Phi}(q_{\alpha/2})
\end{align*}
Use \textbf{standardization} if a gaussian has unknown mean and variance $X \sim N(\mu,\sigma^2)$ to get the quantiles by using Z-tables (standard normal tables).\\

\begin{align*}
\mathbf{P}\left(X\leq t\right)& = \displaystyle \mathbf{P}\left(Z\leq \frac{t-\mu}{\sigma}\right)\\
& = \mathbf{\Phi}\left(\frac{t-\mu}{\sigma}\right)\\
Z &= \frac{X-\mu}{\sigma} \sim N(0,1)\\
q_{\alpha }& = \frac{t-\mu}{\sigma}
\end{align*}